\RequirePackage[l2tabu,orthodox]{nag} % turn on warnings because of bad style
\documentclass[a4paper,11pt,bibtotoc]{scrartcl}
\usepackage{natbib}

\usepackage[utf8]{inputenc}

\usepackage[T1]{fontenc}        % Tries to use Postscript Type 1 Fonts for better rendering
\usepackage{lmodern}            % Provides the Latin Modern Font which offers more glyphs than the default Computer Modern
\usepackage{amsmath} % Provides all mathematical commands
\usepackage{amsfonts} % Provides all mathematical commands

\usepackage{hyperref}           % Provides clickable links in the PDF-document for \ref
\usepackage{grffile}            % Allow you to include images (like graphicx). Usage: \includegraphics{path/to/file}

% Allows to set units
\usepackage[ugly]{units}        % Allows you to type units with correct spacing and font style. Usage: $\unit[100]{m}$ or $\unitfrac[100]{m}{s}$

% Additional packages
\usepackage{url}                % Lets you typeset urls. Usage: \url{http://...}
\usepackage{breakurl}           % Enables linebreaks for urls
\usepackage{xspace}             % Use \xpsace in macros to automatically insert space based on context. Usage: \newcommand{\es}{ESPResSo\xspace}
\usepackage{xcolor}             % Obviously colors. Usage: \color{red} Red text
\usepackage{booktabs}           % Nice rules for tables. Usage \begin{tabular}\toprule ... \midrule ... \bottomrule

% Source code listings
\usepackage{listings}           % Source Code Listings. Usage: \begin{lstlisting}...\end{lstlisting}
\lstloadlanguages{python}       % Default highlighting set to "python"


\begin{document}

\title{GPROF-NN: A neural-network based implementation of the Goddard Profiling Algorithm}
\date{\today}

\maketitle


\section{Introduction}

The Goddard Profiling Algorithm (GPROF, \authorcite{kummerow14},
\yearcite{kummerow14}) is the operational precipitation retrieval algorithm for
the passive microwave (PMW) observations from the satellites of the Global
Precipitation Measurement (GPM, \authorcite{hou14}, \yearcite{hou14}) mission.
The aim of GPM is to provide global measurements of precipitation at a temporal
resolution of 3 hours. It is thus an essential component of the system of global
observations that enables monitoring of the hydrological cycle for the benefit
of science and society.

The development of GPROF was originally motivated by the Tropical Rainfall
Measurement Mission (TRMM,), the precursor of the GPM mission, and thus dates
back almost 30 years \citep{kummerow94, kummerow96}. Since then, the algorithm
has undergone several improvements. Methodologically, the most fundamental
modification has been presented already in \citet{kummerow96}, which introduced
the Bayesian retrieval scheme that is based on a database of observations and
known, corresponding precipitation rates and forms the basis of the algorithm
until today. Following that, improvements were mostly related to and driven by
improvements of the database. While the first version of GPROF still used hand
crafted hydrometeor profiles to generate the database, these were replaced by
profiles from a meso-scale weather model \citep{kummerow96}. An important
improvement was the replacement of the model-based database by an
observationally-generated database for the GPROF 2010 algorithm
\citep{kummerow11, kummerow15}, which helped reduce errors caused by
misrepresentation of atmospheric states in the database. The 2014 version
of GPROF \citet{kummerow15} introduced the first fully-parametric version
of the algorithm that was designed to be applicable to all sensors of
the GPM constellation and became the first version of the operational
PMW precipitation retrieval.

This study revisits the actual retrieval method that forms the basis of the
current GPROF algorithm, which was first introduced in \citet{kummerow96}.
Throughout its evolution, the retrieval database of GPROF has grown to a
considerable size and currently contains around $10^9$ observations. This makes
it ideally suited as training data for a machine-learning based retrieval.
However, most common machine-learning regression techniques are incompatible
with the Bayesian formulation of inverse problems \citep{rodgers00} upon which
the current algorithm is based. \citet{pfreundschuh18} proposed quantile
regression neural networks (QRNNs) as a simple approach to reconcile
inverse-problem theory with neural-network retrievals.

This study presents two novel, neural-network-based implementations of the GPROF
algorithm. The first algorithm, referred to as GPROF-NN 0D, is designed to be
functionally identical to the original GPROF algorithm. Similar to GPROF,
it processes observations pixel-by-pixel and  uses the same inputs and
produces the same outputs as GPROF. The second algorithm, referred GPROF-NN 2D,
also produces the same outputs as GPROF, but processes all observations
simultaneously, thus allowing it to make use of structural information in
the observations. The aim of this study is thus to investigate (1) to
what extent a switching to a functionally equivalent neural-network based
retrieval (GPROF-NN 0D) can help to improve retrievals of precipitation
and the 3D structure of hydrometeors and (2) to what extent structural
information in the retrieval can help improve retrieval accuracy.

\section{Methods and data}

Since its conception, the GPROF algorithm has made use of a retrieval
database consisting of satellite observations and corresponding known
precipitation rates. In this sense, the retrieval method alone may be
viewed as a machine learning method. However, since the distribution
and structure of precipitation was unknown and remains affected by
significant uncertainties, the focus of improvements of the GPROF
algorithm has been generation and representativeness of the database.


This section presents the retrieval database that forms the basis of the
original and the novel GPROF algorithms as well as the data that is used
to test the methods. This is followed by a description of the original
algorithm and the two novel GPROF-NN 0D and GPROF-NN 2D algorithms.

\subsection{The original GPROF algorithm}

The original GPROF algorithm, here referred to simply as GPROF, is a Bayesian
method in the sense that it formulates the retrieval as the task of finding the
posterior distribution $P(\mathbf{x} | \mathbf{y})$ of the precipitation given
an observation vector $\mathbf{y}$. As mentioned above, the posterior
distribution $P(\mathbf{x} | \mathbf{y})$ is calculated using a database
consisting of pairs $(\mathbf{y}_1, \mathbf{x}_1), \ldots (\mathbf{y}_N,
\mathbf{x}_N)$ of observations $\mathbf{y}_i$ and corresponding atmospheric
states $\mathbf{x}_i$. Assuming that the database describes the true a priori
$P(\mathbf{x})$, i.e. the climatological distribution of surface precipitation,
sufficiently well, the expected value w.r.t to the posterior distribution
$P(\mathbf{x} | \mathbf{x})$ of the atmospheric state $\mathbf{x}$ can be
approximated using
\begin{align}\label{eq:gprof_retrieval}
  \int_{\mathbf{x} } \mathbf{x}\: dP(\mathbf{x} | \mathbf{y}) =
  \int_{\mathbf{x} } \mathbf{x}\: \frac{P(\mathbf{y} | \mathbf{x})}{P(\mathbf{y})} dP(\mathbf{x}) \approx
  \frac{\sum_i \mathbf{x} P(\mathbf{y}|\mathbf{x}_i)}{\sum_i P(\mathbf{y}|\mathbf{x})}
\end{align}
The conditional probability $P(\mathbf{y} | \mathbf{x}_i)$ of the observation
$\mathbf{y}$ given an atmospheric state from the database is modeled assuming
  unbiased  Gaussian errors in the observations:
  %
\begin{align}\label{eq:gprof_error}
  P(\mathbf{y}|\mathbf{x}_i) = \frac{1}{\sqrt{2\pi\text{det}(\mathbf{S}^{-1})}}
  \exp \left \{
  - \frac{1}{2}
  (\mathbf{y} - \mathbf{y}_i)^T \mathbf{S}^{-1}  (\mathbf{y} - \mathbf{y}_i)
  \right \}
\end{align}
for a certain covariance matrix $\mathbf{S}$ describing the errors due to noise
as well as other potential error sources that affect the comparison of a real
observation $\mathbf{y}$ and the observations in the database.
Eq.~(\ref{eq:gprof_retrieval}) corresponds to a resampling of the
states in the database with case-specific weights calculated using
Eq.~\ref{eq:gprof_error}. This can be extended to obtain specific quantiles of
the a posteriori distribution (described for example in \citet{pfreundschuh18})
or to derive probabilities of certain characteristics of the a posteriori state
such as the presence of precipitation in a given observation.

A difficulty of the formulation in Eq.~(\ref{eq:gprof_retrieval}) is that
calculating the retrieval result requires traversing a large database. To
speed up the retrieval, observations in the GPROF database are clustered
with respect to the database observations $\mathbf{y}_i$, so that for each
cluster of observations only the central observations, the corresponding
mean state and the number of observations in the cluster need to be stored.

Since the retrieval of complex atmospheric states from the limited information
content of a multi-spectral microwave observation is underconstrained, it is
desirable to incorporate ancillary information to increase its accuracy. In
GPROF the database is split into strata with respect to two-meter temperature
($T_{2m}$), total-column water vapor ($\text{TCWV}$) surface type ($\text{ST}$)
as well as an additional airlifting index for observations in mountain regions.

The basic functioning of the GPROF algorithm is illustrated in
Fig.~\ref{fig:gprof_legacy}. Panel (a) illustrates the reweighting of the
database entries according to Eq.~(\ref{eq:gprof_retrieval}) and
(\ref{eq:gprof_error}). The clustering of the observations to improve retrieval
performance and the stratification of the database to incorporate acillary data
is illustrated in panel (b) and (c), respectively.

\begin{figure}[hbpt]
  \centering
    \includegraphics[width=\textwidth]{../figures/gprof_legacy}
    \caption{The GPROF retrieval algorithm. Panel (a) illustrates the
      resampling of the observation based in the database to approximate the a
      posteriori distribution. Panel (b) exemplifies the clustering of
      observations that is performed to speed up the calculation of retrieval
      results. Panel (c) illustrates the incorporation of ancillary data in
      through application of a stratified retrieval database.}
  \label{fig:gprof_legacy}
\end{figure}

\subsection{Training and test data}

The neural-network-based implementations of the GPROF algorithm that will be
presented below use the same data for training and evaluation as the original
GPROF algorithm. This is done to ensure a fair comparison comparability of the
algorithms against each other as well as functional equivalence.

The main part of the GPROF retrieval database consists of atmospheric profiles
and corresponding observations derived from a year of combined radar-radiometer
observations from the Dual-frequency Precipitation Radar (DPR) and the GPM
Microwave Imager (GMI). These profiles are complemented with gauge-corrected
precipitation estimates derived from ground radar over the US for snow surfaces
and precipitation derived from ERA5 over sea ice.

The available data has been split into three parts: Data from day 6 to the end
of each month are used for training and data from day 4 and 5 are used to
monitor retrieval performance during training and to tune model hyperparameters.
The data from day 1 to 3 of each month is set aside and used for the final
comparison of the three algorithms presented in Sect.~\ref{sec:results}.

The various retrieval variables that are present in the database are listed
in Tab.~\ref{tab:variables}. While surface precipitation is certainly the
primary target of the GPROF algorithm, the observations are sensitive to
a range of additional quantities. Profiles are represented at 28 levels
with a resolution of $500\ \unit{m}$ between $0$ and $10\ \unit{km}$ and
$1 \ \unit{km}$ above that.

\begin{table}
  \caption{Retrieval quantities in the retrieval database}
  \label{tab:variables}
  \centering
  \begin{tabular}{l|r|r}
    Retrieval variable & Unit & Type\\
    \hline
    \hline

    Surface precipitation & $\unit{mm\ h^{-1}}$ & Scalar \\
    Convective precipitation & $\unit{mm\ h^{-1}}$ & Scalar \\
    Cloud water path & $\unit{kg\ m^{-2}}$ & Scalar \\
    Rain water path & $\unit{kg\ m^{-2}}$ & Scalar \\
    Ice water path & $\unit{kg\ m^{-2}}$ & Scalar \\
    Cloud water content & $\unit{g\ m^{-3}}$ & Profile \\
    Rain water content & $\unit{g\ m^{-3}}$ & Profile \\
    Snow water content & $\unit{g\ m^{-3}}$ & Profile \\
    Latent heat & $\unit{K \ h^{-1}}$ & Profile \\
  \end{tabular}
  \end{table}



\subsection{Probabilistic regression with neural networks}

The ill-posed nature of the retrieval of many atmospheric variables from
satellite observations requires a probabilistic approach in order to quantify
the inherent uncertainty of the retrieval results. In many retrievals these
uncertainties are handled by employing Bayesian methods. As shown in a previous
study \cite{pfreundschuh18}, instead of a direct application of Bayes theorem
the posterior distribution may also be predicted directly using neural networks
in a probabilistic regression framework.

The approach proposed in \citet{pfreundschuh18} uses a quantile regression
neural network (QRNN) to predict a sequence of quantiles of the posterior
distribution and has been show to yield equally reliable results as other
explicitly Bayesian retrieval methods that are commonly used to retrieve
atmospheric quantities. For each scalar retrieval variable $x$, the QRNN is
trained to predict a vector of values corresponding to the quantiles of the a
posteriori distribution by minimizing the sum of the quantile losses
\begin{align}
  \mathcal{L}_\tau(\hat{x}, x) &= (\tau - \mathbb{I}_{x < \hat{x}})(x - \hat{x})
\end{align}
for the selected quantile levels $\tau = \tau_1, \ldots, \tau_n$.

In addition to QRNNs, another approach for non-parametric probabilistic
regression has been considered for the GPROF-NN algorithm. This approach
consists of predicting a binned approximation of the probability density
function (PDF) of the posterior distribution. A prior application of this
approach can be found in \cite{metnet}. In lack of a better name, we will refer
to this approach as density regression neural network (DRNN). For each scalar
retrieval variable $x$, the DRNN is trained to classify the retrieval input into
a discrete sequence of bins covering the range of values of $x$. This is done
by converting values of the scalar retrieval variable $x$ into a categorical
variable and training the network using a categorical cross entropy loss:
\begin{align}
  \mathcal{L}(\hat{x}, x) &= \log(p_i) - \sum_{i \neq j}  \log(p_j)
\end{align}

Figure~\ref{fig:qrnn_drnn} illustrates these two approaches when applied to
retrieve surface precipitation along the central pixels of the GMI swath.
While the QRNN provides a direct estimate of the posterior CDF, the DRNN
provides an estimate of the corresponding PDF. Since both approaches provide
a direct estimate of the a posteriori distribution, the two approaches can
be used interchangeably.

\subsection{GPROF-NN 0D}

The GPROF-NN 0D algorithm retrieves all cloud- and precipitation-related
retrieval variables from a single observation and the ancillary data, which
makes it functionally equivalent to that algorithm.

\subsubsection{Network architecture}

\begin{figure}[hbpt]
  \centering
    \includegraphics[width=\textwidth]{../figures/gprof_nn_0d_cdf}
    \caption{Illustration of the neural network architecture used in the
      GPROF-NN 0D algorithm. The network consists of a common body and
      one head for each retrieval variable. Each block in body and head
      consists of fully-connected layer, layer norm and GELU activation function.
    }
  \label{fig:gprof_nn_0d}
\end{figure}

The network used in the GPROF-NN 0D algorithm is a feed-forward multi-layer
perceptron a  separate head for each of the different retrieval variables.
Profile variables share a single head and the expansion of the profile levels
is performed in the last layer of the network. The basic building block of
the network consists of a fully-connected layer followed by layer normalization
and a GELU activation function. The body of the network consists of
$N_b$ of those blocks and each head of $N_h$. Note that the case $N_b = 0$ is
equivalent to predicting each retrieval target with a separate neural network.
A shared body in the neural network allows computations to be shared between
different retrieval variables and therefore makes inference and training more
efficient.

\subsubsection{Training}

The GPROF-NN 0D algorithm is trained to predict all retrieval variables in
parallel by simply minimizing the sum of all losses. Each network is trained for
for 70 epochs using the Adam optimizer and a cosine annealing learning rate
schedule. Warm restarts are performed after 10, 20 and 40 epochs. The initial
learning rates for the first and second part of the training set to $5\cdot
10^{-4}$ and reduced to $10^{-4}$ for its last part.

\subsubsection{Hyperparameter tuning}

To find a well-performing model, a range of configurations of the network
architecture described in Fig.~\ref{fig:gprof_nn_0d} has been trained on roughly
$10\ \unit{\%}$ of the training data and evaluated on a similar fraction of the
validation data. Each model configuration has been trained once as a QRNN and
once as a DRNN. The resulting mean percentage (MPE) error and mean absolute
percentage errors (MAPE) for a subset of the retrieval variables are displayed
in Fig.~\ref{fig:gprof_nn_0d_tuning}.


\subsection{GPROF-NN 2D}

\subsubsection{Architecture}

The network architecture for the GPROF-NN 2D algorithm is a fully-convolutional
asymmetric encoder-decoder architecture followed by a separate network head for
each retrieval variable. Each stage of the encoder consists of a downsampling
layer followed by $N_B$ convolution blocks, each of which consists of two
exception layers followed by layer activation and GELU activation functions. The
input and outputs of each of those blocks are connected by a residual
connection. The decoder stages contain only a single of those blocks per layer.
Additional skip connections are inserted to provide a short cut between
respective stages of the encoder and decoder.

\begin{figure}[hbpt]
  \centering
  \includegraphics[width=0.8\textwidth]{../plots/gprof_nn_2d_arch}
  \caption{
    The neural network architecture employed by the GPROF-NN 2D retrieval.
  }
  \label{fig:data_augmenation}
\end{figure}

\subsubsection{Data augmentation}


\begin{figure}[hbpt]
  \centering
    \includegraphics[width=\textwidth]{../figures/data_augmentation}
    \caption{
      Illustration of the data augmentation applied for the training of the GPROF-NN 2D algorithm.
    }
  \label{fig:data_augmenation}
\end{figure}

Two properties make the application of a convolutional neural network to the GMI
observations peculiar: First, ground truth data is available only at the center
of the input images where the GMI observations overlap with DPR observations.
Secondly, the GMI observations are, due to the conical scanning of the sensor,
not invariant with respect to translations. This means that when the GMI
observations are interpreted as a rectilinear image any observed structure will
look different at the center of the swath compared to the edges. We found that
naive training of a convolutional neural network with this data can lead to
artifacts outside the swath center. To avoid these, the original images in the
training data are augmented on-the-fly with remapped versions that simulate the
effect of the relative placement of the observations in the swath. In addition
to this, observations from channels 10 and up are randomly set to missing in the
outer edges of the swath to simulate the effect of the missing observations that
effects the outer edges of the GMI swath.



\subsubsection{Validation-set performance}


\section{Results}

This section presents the results of the experiments that were performed to
asses the characteristics of the GPROF, GPROF-NN 0D and GPROF-NN 2D retrieval
algorithms. 

\subsection{Results on test dataset}

We begin be assessing the retrieval performance on the test data set, which
consists of the database observations from the first three days of every month.
Since the training data used to train the neural-network retrievals stems from
the same time period, this allows assessing the inherent capabilities of the
retrieval algorithms while minimizing potential aliasing effects caused by the
inter-annual variability of precipitation.



\subsubsection{Scalar retrieval variables}

Retrieval results for scalar retrieval targets are scattered against their
reference values in Fig~\ref{fig:results_scalar}. It should be noted that
frequencies have been normalized column-wise since results for high rain rates
would otherwise not be visible in the plots. As these results show, there is a
consistent, gradual improvement between the GPROF, GPROF-NN 0D and GPROF-NN 2D
algorithms for all retrieval targets. 

The neural-network algorithms generally show the largest improvements for small
values where for surface precipitation and convective precipitation systematic
deviations from the reference distribution are observed. Although improvements
at larger values are generally smaller, they are still consistent across the
examined ranges. The most significant improvement is observed for convective
precipitation. Here the distributions of the GPROF and GPROF-NN 0D algorithms
deviate notably from the diagonal but that of GPROF-NN 2D is located correctly
over the diagonal.


\begin{figure}[hbpt]
  \centering
  \includegraphics[width=\textwidth]{../plots/results_scatter}
  \caption{
    Scatter plots of scalar retrieval targets for the
    three retrieval algorithms. The columns display the results
    for the GPROF, GPROF-NN 0D and GPROF-NN 2D algoriths, while
    rows display the results for different retrieval targets. Frequencies
    in the plots have been normalized column-wise, i.e. per bin of
    the reference value.
  }
  \label{fig:data_augmenation}
\end{figure}

Since the surface type has a strong effect on the lower-frequency observations
used in the retrieval, its impact on the retrieval of surface precipitation is
assessed in Fig.~\ref{fig:results_surface}. The figure displays the Bias, RMSE
and MAPE for the different surface classes that are distinguished in the
retrieval for each of the three algorithms.

Also when considered separately for different surface types the results of the
surface precipitation retrieval show the same overall pattern as observed in
the scatter plots in Fig.~\ref{fig:results_scalar}: The results of the GPROF-NN 0D
retrieval are more accurate than those of the GPROF algorithm and the results
of the GPROF-NN 2D algorithm are in-turn better than those of the GPROF-NN 0D
algorithm. These results are observed consistently across all surface types
and considered metric.


\begin{figure}[hbpt]
  \centering
  \includegraphics[width=\textwidth]{../plots/results_surface_types}
  \caption{
    Bias, root-mean squared error (RMSE), and maximum absolute percentage error (MAPE) for
    the three retrieval algorithms and different surface types.
  }
  \label{fig:data_augmenation}
\end{figure}

In addition to quantitative precipitation estimates, the GPROF algorithm also provides a
probabilistic classification of pixels into raining and non-raining pixels by predicting
a probability of precipitation. As described in ~\ref{sec:method}, a similar quantity can
be derived from the probabilistic prediction of the QRNN-NN algorithms. To assess the quality
of these predictions, the calibration of the predicted probability and the receiver-operating
characteristic curve are displayed in Fig.~\ref{fig:pop}.

The calibration curves show, that predictions from all algorithms are fairly
well calibrated, i.e. the predicted probabilities agree well with those observed
on the independent test data. Although all algorithms exhibit small deviations from the 
diagonal, the GPROF-NN algorithms provide slightly better calibrated predictions.

For the ROC curve, the results are more conclusive and show a clear improvement in classification
skill for the GPROF-NN 0D algorithm compared to GPROF. The GPROF-NN 2D algorithm further improves
the classification slightly compared to the GPROF-NN 0D algorithm.


\begin{figure}[hbpt]
  \centering
  \includegraphics[width=\textwidth]{../plots/results_pop}
  \caption{
    Calibration (Panel (a)) and receiver-operating characteristic (ROC, Panel(b)) for
    the predicted probability of precipitation.
  }
  \label{fig:roc}
\end{figure}

\subsubsection{Profile retrieval variables}

Finally, we also assess the retrieval performance of profile retrievals.
Profiles of Bias and RMSE for the three retrieved profile quantities are
displayed in Fig.~\ref{fig:profiles}. The results show a significant improvement
in bias and RMSE for the GPROF-NN retrievals for all three quantities.
However, there is only a small difference between the GPROF-NN 0D and
GPROF-NN 2D algorithms.


\subsubsection{Probabilistic precipitation retrievals}

The error statistics for the quantitative retrieval targets presented above all
used the mean of the a posteriori distribution to compare against the GPROF
algorithm. Traditionally, GPROF produces only 4 output values that make use of
information from the posterior distribution exceeding the posterior mean: The
estimated probability of precipitation, the precipitation flag, the 1st and 3rd
terciles of the surface precipitation (provided as a measure of uncertainty) and
the most likely precipitation. The neural-network version of GPROF predict the
full posterior distribution for the scalar retrieval variables, and so storing
only the posterior mean discards a large part of this information.

Since storing the full posterior distribution would require excessive amounts
of storage space, we here aim to explore the benefits of providing a random
sample from the posterior distribution as retrieval result. Although the
sampling neglects spatial correlation results, the results should be
able to truthfully reproduce the distribution of precipitation values in the
test data when averaged over multiple orbits.

To illustrate the statistical benefits of randomly-sampled retrieval estimates
we begin by examining the zonal distributions of surface precipitation on
$1^\circ$-degree bins. Fig.~\ref{fig:results_zonal} displays the reference
distribution of the test data (filled contours) together with the quantiles of
the retrieved distributions (shaded lines) for the case when the posterior mean
is used as the retrieval estimate (1st row) and for the case when a random
sample is used as retrieval estimate (2nd row). Note that the color scaling
has been chosen so that if the retrieved and reference distribution were similar,
the shaded lines would lie on top of the boundaries between the filled contours
in the background.

As the first row of panels shows, the retrieved distribution does not match the
corresponding distribution of the test data. Especially the high quantiles of
the reference distribution are strongly underestimated by the retrieved
distributions. However, when instead of the posterior mean random samples are
used as precipitation estimates, the agreement between the retrieved and the
test distribution is significantly improved. Although some discrepancies remain
for the lower quantiles of the distribution, very good agreement is found between
the high quantiles of the distribution.

  \begin{figure}[hbpt]
    \centering
    \includegraphics[width=\textwidth]{../plots/zonal_distributions}
    \caption{
      Zonal distributions of test data and corresponding retrieved values. Colored, filled contours
      in the background show the CDFs of the zonal distribution of precipitation in the test data.
      Grey-shaded line display the quantiles of the corresponding retrieved distributions based on
      the posterior mean (Row 1) and random samples from the posterior distribution (Row 2).
    }
    \label{fig:results_zonal}
  \end{figure}

  For a different perspective on these results, we have also considered the
  ability of the retrieval to reproduce specific percentiles of the test data on
  a spatial grid of size $\SI{5}{\degree}\times\SI{5}{\degree}$. When the
  distribution of percentiles derived from the retrieved values are plotted
  against the percentiles of the test data (Fig.\ \ref{fig:scatter_percentiles})
  it becomes evident that using the posterior mean (grey-shaded, filled
  contours) leads to systematic underestimation of the true value. However, when
  random samples are used these systematic deviations are corrected and the
  percentiles of the retrieved distributions scatter symmetrically around the
  line of equality.

  \begin{figure}[hbpt]
    \centering
    \includegraphics[width=\textwidth]{../plots/scatter_percentiles}
    \caption{
      95th (Row 1) and 99th (Row 2) percentiles of the distribution
      of surface precipitation in $5^\circ \times 5^\circ$ boxes in the
      test data scattered against the corresponding retrieved values. Filled
      contours in the background show the distribution obtained by using the
      posterior mean as retrieved value and colored contours the distribution
      that is obtained when a random sample from the posterior distribution
      is used as retrieval value. The columns show the results for the
      GPROF-NN 0D (Column 1) and the GPROF-NN 2D algorithm  (Column 2).
    }
    \label{fig:scatter_percentiles}
  \end{figure}

\begin{figure}[hbpt]
  \centering
  \includegraphics[width=\textwidth]{../plots/results_spatial}
  \caption{
    Spatial distributions of Bias (Column 1), RMSE (Column 2) and MAPE (Column 3) error statistics for the
    GPROF (Row 1), GPROF-NN 0D (Row 2), GPROF-NN 2D (Row 3) algorithms.
  }
    \label{fig:results_spatial}
\end{figure}


\subsection{Case study: Hurricane Dorian}

To conclude this investigation into the retrieval performance of the newly
developed GPROF retrievals, we consider a overpass of the GPM Core Observatory
over Hurricane Dorian that occurred on September, 1st 2019 at 20:20 UTC.
Since. This allows us to compare the results of the GPROF retrieval to the
GPM combined product at the center of the swath, which has been averaged using
a Gaussian kernel corresponding to a footprint with a full-width at half-maximum
of $\SI{18}{\kilo \meter}$ in along-track and $\SI{10}{\kilo \meter}$ in
across-track direction.

To allow for a detailed comparison, Fig.~\ref{fig:hurricane_precip} displays the
retrieved surface precipitation on a log-scale over a region of $\SI{2000}{\kilo
  \meter}$ as well as on a linear scale on a smaller region close to the
hurricane. When considered at log-scale, obvious structural differences in the
retrieved precipitation are visible: The GPROF retrieval produces large swaths
of low precipitation that cover most of the scene, which are not present in the
combined retrieval. They are clearly reduced in the results of the GPROF-NN 0D
retrieval and even more so in the results of the GPROF-NN 2D retrieval which
agrees well with the results from the combined retrieval.

The two rightmost panels of Fig.~\ref{fig:hurricane_precip} display the the
regions of high precipitation around the eye of the hurricane in more detail.
The most salient features in the surface precipitation from the combined
algorithm is the eye wall and the concentric rain band situated south-west of
the eye. Both the high precipitation withing the eyewall as well as the rainband
are visible in the results of the neural-network versions of GPROF. While the
surface precipitation in the outer rain band is retrieved accurately both
retrievals overestimate the precipitation in the righter flank of the eyewall.
However, compared to the standard GPROF algorithm, the results agree
considerably better with those of the combined algorithm.


\begin{figure}[hbpt]
  \centering
  \includegraphics[width=\textwidth]{../plots/hurricane_surface_precip}
  \caption{
    Retrieved surface precipitation from the GPM combined product (Panels (a),
    (c)), the GPROF algorithm (Panels (b), (d)), and the neural network versions
    GPROF-NN 0D (Panels (e), (g)) and GPROF-NN 2D (Panels (f), (h)) around
    Hurricane Dorian on 2019-09-01 21:23:00 UTC. Panels (a), (b), (e), (f)
    display the precipitation on a logarithmic scale over an area
    spanning $20\ 000\ \unit{km}$ in zonal and meridional directions. Panels (c),
    (d), (g), (f) display the surface precipitation on a linear color scale in
    an enlarged area spanning $5\ 000\ \unit{km}$ around the Hurricane. In the
    background the closest true-color composite from GOES 16 is shown.
  }
  }
  \label{fig:hurricane_precip}
\end{figure}

Figure~\ref{fig:hurricane_3d} displays the retrieved RWC and SWC for the same
observation of Hurricane Dorian. The four panels show isosurfaces the retrieved
water contents from the combined retrieval, GPROF and the GPROF-NN 0D and 2D
algorithms. Also here, the agreement with the results from the combined retrieval
is better for the GPROF-NN retrievals than for GPROF. Especially, the GPROF-NN 2D
agrees well with the combined results.


\begin{figure}[hbpt]
  \centeringn
  \includegraphics[width=\textwidth]{../plots/dorian.r4.0548.png}
  \caption{
    3D isosurfaces of retrieved rain water content (RWC, red) and snow water
    content (SWC, blue) for the same scene as depicted in
    Fig.~\ref{fig:hurricane_precip}. Different panels display the results from
    the GPM combined product, GPROF and the GPROF-NN 0D and 2D algorithms.
  }
    \label{fig:hurricane_3d}
\end{figure}


\section{Discussion}

Above we have presented two neural network based implementations of the GPROF
algorithm. Both algorithms were designed to be functionally equivalent to the
current implementation which means that they can potentially replace the current
method in an upcoming version. We have assessed the performance of the three
algorithms on test data that was not seen by any of the neural network based
retrieval methods before the evaluation. Here we have seen a significant and
consistent improvements of the retrieval accuracy across all considered metrics
and outputs. Similar tendencies were found when comparing retrieval results from
a Hurricane overpass outside the database period to the GPM combined retrieval.

\subsection{Retrieval performance}

Our results show a consistent improvement of the GPROF-NN retrievals over the
original GPROF algorithm. This is of course an encouraging result considering
that all three retrievals are based on exactly the same retrieval database,
indicating that the GPROF retrievals can be improved simply by switching to a
more powerful method.

But why do the neural-network-based retrievals perform better than the
traditional approach? To investigate this, we have calculated the gradients of
the retrieved surface precipitation with respect to the brightness temperatures
for the GPROF and GPROF-NN 0D retrievals. Comparison of the mean gradients and
their component-wise standard deviation reveals that the GPROF-NN 0D retrieval
exhibits larger gradients with more variation than the GPROF retrieval. This
indicates that the retrieval manifold that describes the relation between input
observations and retrieved surface precipitation exhibits more structure for the
neural network retrieval than it does for standard GPROF.

Based on the formulation of the standard GPROF retrieval, the retrieval may be
viewed as nearest-neighbor regression with a Gaussian kernel and a diagonal
covariance matrix. Although the theoretical formulation of the retrieval assumes
the elements of the covariance matrix to correspond to the standard deviations
of the random errors in the observations, these errors are in practice inflated
to account for sparsity of the retrieval database. Our results suggest that this
may limit the ability of the method to reproduce structure of the retrieval
manifold leading to a loss of sensitivity.

We find additional improvements in retrieval accuracy when structural
information is included in the retrieval by means of processing scenes of
observations simultaneously instead of pixel-by-pixel. Intuitively, this makes
sense since a human observer with some experience with microwave imagery can
identify characteristic rain structures. An interesting continuation of
this work would be to investigate, which structural patterns the neural
network has learned, which we aim to pursue in a follow-up study.

\subsection{The importance of ancillary data}

An operational aspect that considerably complicates the GPROF algorithm is
the required ancillary data. Currently two version of GPROF exist: The
near real time version, which uses forecasts from the GANAL model as source
for the ancillary data, and a climate version, which uses ERA5 data from
\citep{era5}. For the development of the GPROF-NN algorithms, we have
investigated how much the inclusion of ancillary actually improves the
retrieval. The results are presented in the appendix. Although omitting
the ancillary data has a negative impact on the retrieval performance,
both retrievals still perform better than the original GPROF
implementation. Moreover, the GPROF-NN 2D retrieval without ancillary
data performs only marginally worse than the retrieval with ancillary
data and better than both variants of the GPROF-NN 0D retrieval.

Although this may not have any direct consequences on the operational
application of GPROF, it still has interesting consequences for the development
of future retrievals. For one, this obviously means that incorporating spatial
information into a MW precipitation retrieval improves the retrieval to a larger
extent than adding ancillary information from secondary sources without adding
latency due to data availability constraints. But, this also opens up the
possibility of distributing the retrieval algorithm instead of the retrieval.
The neural network model for the GPROF-NN 2D algorithm has a size of about
$\SI{70}{\mega \byte}$, which makes it roughly as large as the level 1C file
required to run the retrieval. This use case may be interesting for the
development of specialized retrievals that have a lower general interest but
could in this way be made available to the community without the need to set up
an operational processing pipline. An example would be a high-resolution
hydrometeor profile, which may be of interest for case studies but whose output
would be too large to archive operationally.

\subsection{Limitations}

Our main finding is that the accuracy of the GPROF retrieval can be
significantly improved by replacing the retrieval method with a more expressive
one. However, The majority of the experiments on which this finding is based use
data from the time period covered by the retrieval database. Based on these
results alone it is therefor not possible conclude that these models will
generalize to other time periods. This is why, as a next step, we will test the
both neural-network based retrieval algorithms in an operational setting to see
how they perform along side the standard GPROF retrieval.

Another aspect that has been neglected in this study is the accuracy of the
retrieval database that is used both in standard GPROF and the neural-network
version. The database is derived from the GPM combined product combined with
MIRS, ERA5 and MRMS. Its accuracy will thus always be limited by the accuracy of
those products. To what degree the improvements in the GPROF retrieval translate
to independent validation data thus remains to be analyzed in a future study.

Finally, another point that should be mentioned here is that although both
GPROF-NN are porbabilistic in the sense that they provide an estimate of
the postior distribution, they model the distributions of the output values
as independent distributions. This means that, although uncertainties are
well calibrated on average, they may not be on the scale of single
precipitation systems. However, the is also the case for the standard
GPROF retrieval and the authors are note aware of any method that could solve
this issue in ana operational context.

\section{Conclusions}

The GPROF-NN algorithms and their performance assessments that were presented in
this study constitute a first step towards a neural network based implementation
of the passive precipitation retrievals for the GPM constellation. Although it
has to be taken into account that our experiments neglect relevant errors
sources that will ultimately add to the full retrieval uncertainty, our results
remain encouraging: Simply by replacing the retrieval with a neural network
based method the accuracy of the retrieval can be increased at the same time as
the computational complexity of the retrieval is decreased. This potentially
makes the upgrade of the retrieval implementation a very simple and cheap way
of improving global precipitation measurements.

As a next step we will run extend the implementation of the GPROF-NN 0D and
GPROF-NN 2D algorithms to the other sensors of the GPM constellation and run
them operationally along side standard GPROF. The three methods will then
be validated and assessed using gauge-corrected gound-based radar measurements
in a follow up study. This is to ensure that the upgrade of the retrieval method
doesn't degrade the quality of the operational GPM PMW retrievals.

In addition to potential improvements in retrieval accuracy through an upgrade
of the retrieval method, we presented two novel applications that would be
enabled by the neural-network based retrievals:

\begin{enumerate}
\item The inclusion of random samples from the posterior distribution in the
  retrieval output: We have presented this as an efficient way of representing
  retrieval uncertainties that helps to improve the representation of extreme
  events in the retrieval results.
\item Running retrieval 'on the edge': Our results show that incorporating
  structural information into the retrieval yields larger performance benefits
  than incorporating ancillary information. This allows simplifying the
  retrieval pipeline to the point that only observations are required as input,
  which, together with the availability of high-performance machine learning
  software, opens up the possibility letting the user run the retrieval on his
  own computer, i.e. to perform the inference 'on the edge'. This may be
  beneficial for retrievals with very high resolution, i.e. whose storage
  requirement are much larger than those of the L1C data, and which may be
  interesting to users on a case-by-case basis.
\end{enumerate}

Building on the approach proposed in \citet{pfreundschuh18}, this work presents
a way of integrating and reconciling machine learning methods with retrieval
pipelines built on Bayesian retrieval algorithms. We find that the machine
learning based retrieval, which is built on readily available, high-performance
machine learning libraries, is easiert to implement, runs faster and performs
better than the hand-crafted retrieval algorithm. Although it remains to be seen
to which extent this will improve retrievals with respect to independent
validation data, it seems reasonable to expect at least incremental improvements
in global precipitation measurements as well as interesting opportunities for
the novel retrieval applications.


\bibliographystyle{alpha}
\bibliography{research}

\end{document}
